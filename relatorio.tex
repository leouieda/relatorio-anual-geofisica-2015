\documentclass[12pt,a4paper]{article}
\usepackage[utf8]{inputenc}
\usepackage[T3,T1]{fontenc}
\usepackage[portuges]{babel}
\usepackage[noenc]{tipa}
\usepackage{tipx}
\usepackage[geometry,weather,misc,clock]{ifsym}
\usepackage{pifont}
\usepackage{eurosym}
\usepackage{amsmath}
\usepackage{wasysym}
\usepackage{amssymb,amsfonts,textcomp}
\usepackage{color}
\usepackage{array}
\usepackage{hhline}
\usepackage{hyperref}
\hypersetup{pdftex, colorlinks=true, linkcolor=blue, citecolor=blue,
            filecolor=blue, urlcolor=blue, pdftitle=}
\usepackage[pdftex]{graphicx}

% Page layout (geometry)
\setlength\voffset{-1in}
\setlength\hoffset{-1in}
\setlength\topmargin{0.7874in}
\setlength\oddsidemargin{0.9846in}
\setlength\textheight{10.118099in}
\setlength\textwidth{6.2988997in}
\setlength\footskip{0.0cm}
\setlength\headheight{0cm}
\setlength\headsep{0cm}

% Pages styles
\makeatletter
\newcommand\ps@Standard{
  \renewcommand\@oddhead{}
  \renewcommand\@evenhead{}
  \renewcommand\@oddfoot{}
  \renewcommand\@evenfoot{}
  \renewcommand\thepage{\arabic{page}}
}
\makeatother
\pagestyle{Standard}
\setlength\tabcolsep{1mm}
\renewcommand\arraystretch{1.3}


\begin{document}

\begin{tabular}{lcr}
    \includegraphics{img/logo-on.jpg}
    & \hspace{5cm} &
    \includegraphics{img/logo-mcti.png}
\end{tabular}

\vspace{0.5cm}

\begin{center}
    \textbf{\LARGE RELATÓRIO 20XX}

    \vspace{1cm}

    \textbf{TÍTULO DO PROJETO DE TESE OU DISSERTAÇÃO}
\end{center}

\vspace{1.5cm}

\begin{flushleft}

{\bfseries
NOME DO ALUNO:
}

{\bfseries
NIVEL:
}

{\bfseries
ENDEREÇO RESIDENCIAL:
}

{\bfseries
TELEFONES FIXO e CELULAR:
}

{\bfseries
EMAIL NÃO INSTITUCIONAL:
}

{\bfseries
DATA DE INÍCIO NA PÓS-GRADUAÇÃO:
}

{\bfseries
DATA DE INÍCIO DO PROJETO DE PESQUISA:
}

{\bfseries
NOME DO ORIENTADOR:
}

{\bfseries
PERÍODO PREVISTO DE BOLSA DE ESTUDOS:
}

{\bfseries
PERÍODO A QUE SE REFERE O RELATÓRIO:
}

\vfill

{\small Este relatório foi entregue ao orientador no dia}
\rule{1.5cm}{0.4pt}
\hfill
\rule{4cm}{0.4pt}

\vspace{-0.2cm}

\hfill {\tiny ASSINATURA DO ORIENTADOR}

\vspace{0.5cm}

{\small O orientador emitiu parecer no dia}
\rule{1.5cm}{0.4pt}
\hfill
\rule{4cm}{0.4pt}

\vspace{-0.2cm}
\hfill {\tiny ASSINATURA DO ORIENTADOR}

\vspace{0.5cm}

{\small Este relatório e o parecer do orientador foi recebido na
SECRETARIA DA PÓS-GRADUAÇÃO no dia:}

\vspace{0.5cm}

\rule{3cm}{0.4pt}
\hfill
\rule{10cm}{0.4pt}

\vspace{-0.2cm}

\hfill {\tiny NOME E ASSINATURA DO FUNCIONÁRIO DA SECRETARIA}

\end{flushleft}

\newpage

%-----------------------------------------------------------------------------
% PAGINA DE INFORMACOES CURRICULARES
%-----------------------------------------------------------------------------
\begin{center}
\textbf{\large INFORMAÇÕES CURRICULARES, CONTENDO OS SEGUINTES ITENS:}
\end{center}

\vspace{1cm}

\noindent (1) TOTAL DE CRÉDITOS CURSADOS EM DISCIPLINAS: \rule{2cm}{0.4pt}

\bigskip

\noindent (2) LISTA DE TODAS AS DISCIPLINAS CURSADAS ATÉ O MOMENTO PELO ALUNO
E OS CONCEITOS OBTIDOS

\bigskip


\bigskip

\noindent (3) SITUAÇÃO DO ALUNO QUANTO AOS CRÉDITOS OBTIDOS NOS SEMINÁRIOS
ANUAIS

\bigskip

{\centering\itshape
Aprovado no(s) ano(s) 20XX e/ou Reprovado no(s) ano(s) 20XX
\par}

\bigskip

\noindent (4) SITUAÇÃO DO ALUNO QUANTO AO EXAME DE PROFICIÊNCIA

\bigskip

{\centering\itshape
Aprovado ou Reprovado
\par}

\bigskip

\noindent (5) SITUAÇÃO DO ALUNO DE DOUTORADO QUANTO AO EXAME DE QUALIFICAÇÃO

\bigskip

{\centering\itshape
Aprovado no \ ano 20XX
\par}

\bigskip

\noindent (6) LISTA DAS REUNIÕES CIENTÍFICAS EM QUE PARTICIPOU NO PERÍODO A QUE
SE REFERE O RELATÓRIO, COM O TÍTULO E AUTORES DOS TRABALHOS APRESENTADOS

\bigskip


\bigskip

\noindent (7) LISTA DOS ARTIGOS PUBLICADOS, ACEITOS OU SUBMETIDOS

\bigskip


\bigskip

\noindent (8) OUTRAS ATIVIDADES RELEVANTES NO PERÍODO (PARTICIPAÇÃO EM
TRABALHOS DE CAMPO, ESCOLAS ETC)

\bigskip


\bigskip

{\centering
(TAMANHO LIVRE)
\par}


\newpage

%-----------------------------------------------------------------------------
% PROJETO
%-----------------------------------------------------------------------------
\begin{center}\textbf{\large PROJETO ORIGINAL}\end{center}

\vspace{1cm}

\textit{
Nesta parte do relatório o estudante deve incluir o projeto de pesquisa tal
como apresentado à Comissão de Pós-Graduação em Geofísica do Observatório
Nacional na época da inscrição.
}

\newpage

%-----------------------------------------------------------------------------
% DESENVOLVIMENTO DO PROJETO DE PESQUISA NO ÚLTIMO RELATÓRIO
%-----------------------------------------------------------------------------
\begin{center}
\textbf{\large RELATO DO DESENVOLVIMENTO DO PROJETO DE PESQUISA NO ÚLTIMO
RELATÓRIO}
\end{center}

\vspace{1cm}

\textit{
Nesta parte do relatório o estudante deve apresentar um resumo do
desenvolvimento do projeto de pesquisa um ano atrás (último relatório)
}

\vspace{1cm}

{\centering
(NO MÁXIMO UMA PÁGINA)
\par}

\newpage

%-----------------------------------------------------------------------------
% TRABALHO DE PESQUISA DESENVOLVIDO NO PERÍODO
%-----------------------------------------------------------------------------
\begin{center}
\textbf{\large DESCRIÇÃO DETALHADA DO TRABALHO DE PESQUISA DESENVOLVIDO NO
PERÍODO DO RELATÓRIO}
\end{center}

\vspace{1cm}

\textit{
Esta parte do relatório deve conter \textbf{obrigatoriamente}
os seguintes itens:
}

\vspace{1cm}

{\centering\bfseries\itshape
Metodologia aplicada ou desenvolvida
\par}

\bigskip

{\centering\bfseries\itshape
Resultados parciais já obtidos
\par}

\bigskip

{\centering\bfseries\itshape
Dificuldades encontradas e como elas estão sendo superadas
\par}

\bigskip

{\centering\bfseries\itshape
Bibliografia utilizada no contexto do trabalho.
\par}

\bigskip

\bigskip

{\itshape
Recomenda-se que o estudante escreva, nesta parte do relatório, textos que
poderão ser aproveitados em suas teses ou dissertações.}

\bigskip

{\itshape
Por exemplo, aqueles alunos que realizaram trabalhos de campo ou de laboratório
no último ano e estes trabalhos fazem parte de seu projeto de pesquisa; então,
estes alunos podem aproveitar para escrever sobre estes trabalhos executados.}

\bigskip

{\itshape
Outro exemplo, aqueles alunos que estão desenvolvendo metodologias ou
aplicações de metodologias existente; então, estes alunos podem aproveitar
para escrever sobre estas metodologias.}

\bigskip

{\itshape
Outro exemplo, aqueles alunos mais adiantados que já escreveram nos relatórios
anteriores os itens acima; então, estes alunos podem aproveitar para escrever
sobre seus resultados apresentando figuras que serão aproveitadas na Tese ou
Dissertação.}

\bigskip

{\itshape
Alguns alunos também podem escrever o texto da introdução do seu trabalho. Em
geral, uma introdução deve: 1) apresentar a natureza e o alcance do problema;
2) revisar a literatura;, 3) apresentar os objetivos do trabalho; 4) descrever
o método de investigação; e 5) descrever os principais resultados da
investigação.}

\vspace{1.5cm}

{\centering
(TAMANHO LIVRE)
\par}

\newpage

%-----------------------------------------------------------------------------
% PRÓXIMAS ETAPAS DO TRABALHO DE PESQUISA
%-----------------------------------------------------------------------------
\begin{center}
\textbf{\large PRÓXIMAS ETAPAS DO TRABALHO DE PESQUISA}
\end{center}

\vspace{1cm}

\textit{Esta parte do relatório deve conter \textbf{obrigatoriamente}
os seguintes itens:}

\bigskip

\bigskip

{\centering\bfseries\itshape
Atividades de pesquisa previstas para o próximo período.
\par}

\bigskip

{\centering\bfseries\itshape
Atividades acadêmicas previstas para o próximo período.
\par}

\bigskip

{\centering\bfseries\itshape
Cronograma detalhado das atividades
\par}

\bigskip

{\centering\bfseries\itshape
Data prevista de conclusão do mestrado ou doutorado (*)
\par}

\bigskip

\bigskip

{\centering\bfseries\itshape
(*) Se houver atraso (ou previsão de atraso) na finalização da tese ou
dissertação, justifique os motivos do atraso.
\par}

\bigskip
\bigskip

{\centering
(NO MÁXIMO DUAS PÁGINAS)
\par}

\end{document}
