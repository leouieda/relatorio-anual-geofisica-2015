\documentclass[12pt,a4paper]{article}
\usepackage[utf8]{inputenc}
\usepackage[T3,T1]{fontenc}
\usepackage[portuges]{babel}
\usepackage[noenc]{tipa}
\usepackage{tipx}
\usepackage[geometry,weather,misc,clock]{ifsym}
\usepackage{natbib}
\usepackage{hyperref}
\usepackage{pifont}
\usepackage{eurosym}
\usepackage{amsmath}
\usepackage{wasysym}
\usepackage{amssymb,amsfonts,textcomp}
\usepackage{color}
\usepackage{array}
\usepackage{hhline}
\usepackage{booktabs}
\usepackage{hyperref}
\hypersetup{pdftex, colorlinks=true, linkcolor=blue, citecolor=blue,
            filecolor=blue, urlcolor=blue, pdftitle=}
\usepackage{graphicx}
\usepackage{pdfpages}

% Page layout (geometry)
\setlength\voffset{-1in}
\setlength\hoffset{-1in}
\setlength\topmargin{0.7874in}
\setlength\oddsidemargin{0.9846in}
\setlength\textheight{10.118099in}
\setlength\textwidth{6.2988997in}
\setlength\footskip{0.0cm}
\setlength\headheight{0cm}
\setlength\headsep{0cm}

% Pages styles
\makeatletter
\newcommand\ps@Standard{
  \renewcommand\@oddhead{}
  \renewcommand\@evenhead{}
  \renewcommand\@oddfoot{}
  \renewcommand\@evenfoot{}
  \renewcommand\thepage{\arabic{page}}
}
\makeatother
\pagestyle{Standard}
\setlength\tabcolsep{1mm}
\renewcommand\arraystretch{1.3}


\begin{document}

\begin{tabular}{lcr}
    \includegraphics{img/logo-on.jpg}
    & \hspace{5cm} &
    \includegraphics{img/logo-mcti.png}
\end{tabular}

\vspace{0.5cm}

\begin{center}
    \textbf{\LARGE RELATÓRIO 2015}

    \vspace{1cm}

    \textbf{MODELAGEM E INVERSÃO DE CAMPOS GRAVITACIONAIS EM COORDENADAS
            ESFÉRICAS}
\end{center}

\vspace{1.5cm}

\begin{flushleft}

{\bfseries
NOME DO ALUNO:
}
Leonardo Uieda

{\bfseries
NIVEL:
}
Doutorado

{\bfseries
ENDEREÇO RESIDENCIAL:
}
Rua Aguiar, 49 ap 104. 20261-120 Tijuca, Rio de Janeiro - RJ

{\bfseries
TELEFONES FIXO e CELULAR:
}
(21) 983636761

{\bfseries
EMAIL NÃO INSTITUCIONAL:
}
leouieda@gmail.com

{\bfseries
DATA DE INÍCIO NA PÓS-GRADUAÇÃO:
}
11/2011

{\bfseries
DATA DE INÍCIO DO PROJETO DE PESQUISA:
}
11/2011

{\bfseries
NOME DO ORIENTADOR:
}
Valéria C. F. Barbosa

{\bfseries
PERÍODO PREVISTO DE BOLSA DE ESTUDOS:
}
11/2011 - 10/2015

{\bfseries
PERÍODO A QUE SE REFERE O RELATÓRIO:
}
03/2014 - 02/2015

\vfill

{\small Este relatório foi entregue ao orientador no dia}
\rule{1.5cm}{0.4pt}
\hfill
\rule{4cm}{0.4pt}

\vspace{-0.2cm}

\hfill {\tiny ASSINATURA DO ORIENTADOR}

\vspace{0.5cm}

{\small O orientador emitiu parecer no dia}
\rule{1.5cm}{0.4pt}
\hfill
\rule{4cm}{0.4pt}

\vspace{-0.2cm}
\hfill {\tiny ASSINATURA DO ORIENTADOR}

\vspace{0.5cm}

{\small Este relatório e o parecer do orientador foi recebido na
SECRETARIA DA PÓS-GRADUAÇÃO no dia:}

\vspace{0.5cm}

\rule{3cm}{0.4pt}
\hfill
\rule{10cm}{0.4pt}

\vspace{-0.2cm}

\hfill {\tiny NOME E ASSINATURA DO FUNCIONÁRIO DA SECRETARIA}

\end{flushleft}

\newpage

%-----------------------------------------------------------------------------
% PAGINA DE INFORMACOES CURRICULARES
%-----------------------------------------------------------------------------
\begin{center}
\textbf{\large INFORMAÇÕES CURRICULARES, CONTENDO OS SEGUINTES ITENS:}
\end{center}

\vspace{1cm}

\begin{flushleft}

\noindent (1) TOTAL DE CRÉDITOS CURSADOS EM DISCIPLINAS: 12

\bigskip

\noindent (2) LISTA DE TODAS AS DISCIPLINAS CURSADAS ATÉ O MOMENTO PELO ALUNO
E OS CONCEITOS OBTIDOS

\bigskip

Tópicos de interpretação de dados gravimétricos e magnéticos - A\\
Minicurso: ``Electromagnetic methods in applied geophysics'' - (curso sem
conceito)\\
Fenômenos críticos em geociências - A\\
Inversão em métodos potenciais - A\\
Sísmica aplicada (ênfase em exploração de petróleo e gás) - A

\bigskip

\noindent (3) SITUAÇÃO DO ALUNO QUANTO AOS CRÉDITOS OBTIDOS NOS SEMINÁRIOS
ANUAIS

\bigskip

Aprovado nos anos de 2012 e 2014.

\bigskip

\noindent (4) SITUAÇÃO DO ALUNO QUANTO AO EXAME DE PROFICIÊNCIA

\bigskip

Aprovado.

\bigskip

\noindent (5) SITUAÇÃO DO ALUNO DE DOUTORADO QUANTO AO EXAME DE QUALIFICAÇÃO

\bigskip

Aprovado no ano 2013.

\bigskip

\noindent (6) LISTA DAS REUNIÕES CIENTÍFICAS EM QUE PARTICIPOU NO PERÍODO A QUE
SE REFERE O RELATÓRIO, COM O TÍTULO E AUTORES DOS TRABALHOS APRESENTADOS

\bigskip

Evento: Python in Science Conference 2014\\
Autores: Uieda, L., V. C. Oliveira Jr and V. C. F. Barbosa\\
Título: Using Fatiando a Terra to solve inverse problems in geophysics\\
Material disponível em
\href{https://github.com/leouieda/scipy2014}{https://github.com/leouieda/scipy2014}

\bigskip

Evento: EGU General Assembly 2014\\
Autores: Uieda, L. and V. C. F. Barbosa\\
Título: Gravity inversion in spherical coordinates using tesseroids\\
Material disponível em
\href{https://github.com/leouieda/egu2014}{https://github.com/leouieda/egu2014}

\bigskip

\noindent (7) LISTA DOS ARTIGOS PUBLICADOS, ACEITOS OU SUBMETIDOS

\bigskip

(Submetido) Carlos, D. U., Uieda, L., and V. C. F. Barbosa (2015),
How two gravity-gradient inversion methods can be used to reveal different
geologic features of ore deposit - a case study from the Quadrilátero
Ferrífero (Brazil),
Journal of Applied Geophysics.

\bigskip

(Em revisão) Oliveira Jr, V. C., D. P. Sales, V. C. F. Barbosa, and L. Uieda (2014),
Estimation of the total magnetization direction of approximately spherical
bodies, Nonlinear Processes in Geophysics.

\bigskip

(Publicado) Uieda, L., V. C. Oliveira Jr, and V. C. F. Barbosa (2014), Geophysical
tutorial: Euler deconvolution of potential-field data, The Leading Edge, 33(4),
448-450, doi:10.1190/tle33040448.1.

\bigskip

(Publicado) Carlos, D. U., L. Uieda, and V. C. F. Barbosa (2014), Imaging iron ore from the
Quadrilátero Ferrífero (Brazil) using geophysical inversion and drill hole
data, Ore Geology Reviews, 61, 268-285, doi:10.1016/j.oregeorev.2014.02.011.

\bigskip

\noindent (8) OUTRAS ATIVIDADES RELEVANTES NO PERÍODO (PARTICIPAÇÃO EM
TRABALHOS DE CAMPO, ESCOLAS ETC)

\bigskip

Ministrei o curso "Tópicos de inversão em geofísica" com duração de dois dias
na III Semana de Geofísica da UNB.
O material didático e informações sobre o curso estão disponíveis em
\href{https://github.com/pinga-lab/inversao-unb-2014}{https://github.com/pinga-lab/inversao-unb-2014}


\bigskip

\end{flushleft}


\newpage

%-----------------------------------------------------------------------------
% PROJETO
%-----------------------------------------------------------------------------
\begin{center}\textbf{\large PROJETO ORIGINAL}\end{center}

\vspace{1cm}

\textit{
Nesta parte do relatório o estudante deve incluir o projeto de pesquisa tal
como apresentado à Comissão de Pós-Graduação em Geofísica do Observatório
Nacional na época da inscrição.
}

\includepdf[pages={-},offset=2.5cm -2.5cm]{projeto.pdf}


\newpage

%-----------------------------------------------------------------------------
% DESENVOLVIMENTO DO PROJETO DE PESQUISA NO ÚLTIMO RELATÓRIO
%-----------------------------------------------------------------------------
\begin{center}
\textbf{\large RELATO DO DESENVOLVIMENTO DO PROJETO DE PESQUISA NO ÚLTIMO
RELATÓRIO}
\end{center}

\vspace{1cm}

No período entre Março de 2013 e Fevereiro de 2014,
eu desenvolvi a infraestrutura computacional que seria necessária para a
implementação da inversão de dados gravimétricos em coordenadas esférias.

A primeira etapa foi o desenvolvimento de um conjunto
de classes e funções em linguagem Python para
automatizar a criação de métodos de inversão.
Esse código foi incluído na biblioteca Fatiando a Terra (
\href{http://fatiando.org/}{http://fatiando.org/}),
onde é utilizado para implementar os diversos métodos de inversão já incluídos
na biblioteca.
Esse novo módulo de inversão automatiza as tarefas de
(1) construção da função objetivo,
(2) implementação das diversas formas de regularização,
(3) minimização da função objetivo por diferentes métodos,
(4) cálculo do vetor gradiente e matriz Hessiana da função objetivo
e (5) determinação do parâmetro de regularização através da análise da curva L.

A segunda etapa foi a implementação da modelagem direta em coordenadas
esféricas utilizando tesseroides (prismas esféricos).
Fiz uma implementação da modelagem em linguagem Python para comparar com a
implementação anterior em linguagem C do programa Tesseroids
(\href{http://tesseroids.leouieda.com}{http://tesseroids.leouieda.com}).
Obtive resultados preliminares da análise do erro cometido na integração
numérica do campo gravitacional de um tesseroide.
Existem poucos modelos em coordenadas esféricas com soluções analíticas que
possam ser utilizados na quantificação do erro.
Utilizei o modelo de meia casca esférica, para a qual existe uma solução
analítica para o ponto de observação localizado no polo.
Outro modelo utilizado foi o de um prisma retangular reto.
Este não é um modelo ideal pois a aproximação de um tesseroide por um prisma
retangular piora quanto maior o tesseroide ou quanto mais longe da região
equatorial.



\newpage

%-----------------------------------------------------------------------------
% TRABALHO DE PESQUISA DESENVOLVIDO NO PERÍODO
%-----------------------------------------------------------------------------
\begin{center}
\textbf{\large DESCRIÇÃO DETALHADA DO TRABALHO DE PESQUISA DESENVOLVIDO NO
PERÍODO DO RELATÓRIO}
\end{center}

\vspace{1cm}

{\centering\bfseries\itshape
Metodologia aplicada ou desenvolvida
e
Resultados parciais já obtidos
\par}

\bigskip

Abaixo estão a introdução, metodologia e resultados do artigo sobre a
modelagem direta utilizando tesseroides.
Este trabalho descreve a metodologia para realizar a modelagem e sua
implementação computacional.
A metodologia sofreu diversas modificações durante a escrita do artigo
no ano de 2014.

Também realizei uma implementação inicial da inversão de dados gravimétricos
para determinar o relevo de uma interface.
A parametrização da interface é feita com tesseroides, possibilitando a
inversão em coordenadas esféricas de grandes áreas.
A inversão é feita com uma adaptação do método de \citet{Silva2014}.
A implementação foi feita utilizando o módulo de inversão do software
Fatiando a Terra (descrito no relatório de 2014).
Essa infraestrutura computacional permite a implementação de uma inversão
relativamente complexa em poucas (\~30) linhas de código.
A implementação desta inversão ainda necessita ser testada extensivamente
para verificar sua eficácia.

\bigskip

{\centering\bfseries\itshape
Dificuldades encontradas e como elas estão sendo superadas
\par}

\bigskip

A metodologia para modelagem direta sofreu diversas alterações.
O algoritmo de discretização adaptativa dos tesseroides foi convertido de um
algoritmo recursivo para um algoritmo utilizando uma estrutura de dados
denominada ``pilha'' (mais detalhes em ``Implementation'' abaixo).
Essa mudança aumenta a velocidade do algoritmo pois evita o tempo gasto com
chamadas a funções.

A análise do erro da integração numérica passou a utilizar como referência uma
casca esférica no lugar da meia casca esférica que foi utilizada anteriormente.
Fizemos essa modificação pois a meia casca  esférica possui solução
analítica em um único ponto, no polo.
A figura~\ref{fig:glqerrorsample} mostra que o  maior erro na integração
numérica está em uma área acima  do tesseroide.
Logo, o cálculo em um único ponto  pode não  registrar o maior erro da
integração, subestimando o tamanho do erro
(mais detalhes em ``Evaluation of the accuracy'' abaixo).

\bigskip
\bigskip

{\centering\bfseries\itshape
Partes do artigo sobre modelagem direta com tesseroides
\par}


\input{parte-paper.tex}

\bibliographystyle{seg}
\bibliography{references.bib}

\newpage

%-----------------------------------------------------------------------------
% PRÓXIMAS ETAPAS DO TRABALHO DE PESQUISA
%-----------------------------------------------------------------------------
\begin{center}
\textbf{\large PRÓXIMAS ETAPAS DO TRABALHO DE PESQUISA}
\end{center}

\vspace{0.5cm}

\begin{flushleft}

\textbf{Atividades de pesquisa previstas para o próximo período:}

\begin{enumerate}
    \item Finalização da escrita do artigo sobre a modelagem direta utilizando
        tesseroides. Submissão para a revista \textit{Geophysics}.
    \item Aplicação a dados sintéticos e dados reais da inversão para
        determinação do relevo de interfaces em coordenadas esféricas.
    \item Escrita e submissão do artigo sobre a inversão de interfaces.
    \item Elaboração da tese baseada nos dois artigos.
    \item Defesa.
\end{enumerate}

\textbf{Atividades acadêmicas previstas para o próximo período:}

\bigskip

Nenhuma.

\bigskip

\textbf{Cronograma detalhado das atividades:}

\begin{center}
\begin{tabular}{l|l}
    \toprule
    \textbf{Atividade} & \textbf{Período de execução} \\
    \midrule
    Finalização do artigo de modelagem direta & Março\\
    Finalização da implementação da inversão de interfaces & Março\\
    Aplicação da inversão de interfaces & Abril-Maio\\
    Escrita do artigo de inversão de interfaces & Abril-Agosto\\
    Finalização da tese & Agosto-Setembro\\
    Defesa & Outubro\\
    \bottomrule
\end{tabular}
\end{center}


\bigskip

\textbf{Data prevista de conclusão do mestrado ou doutorado:} Outubro/2015

\end{flushleft}


\end{document}
